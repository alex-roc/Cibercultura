\PassOptionsToPackage{unicode=true}{hyperref} % options for packages loaded elsewhere
\PassOptionsToPackage{hyphens}{url}
%
\documentclass[]{book}
\usepackage{lmodern}
\usepackage{amssymb,amsmath}
\usepackage{ifxetex,ifluatex}
\usepackage{fixltx2e} % provides \textsubscript
\ifnum 0\ifxetex 1\fi\ifluatex 1\fi=0 % if pdftex
  \usepackage[T1]{fontenc}
  \usepackage[utf8]{inputenc}
  \usepackage{textcomp} % provides euro and other symbols
\else % if luatex or xelatex
  \usepackage{unicode-math}
  \defaultfontfeatures{Ligatures=TeX,Scale=MatchLowercase}
\fi
% use upquote if available, for straight quotes in verbatim environments
\IfFileExists{upquote.sty}{\usepackage{upquote}}{}
% use microtype if available
\IfFileExists{microtype.sty}{%
\usepackage[]{microtype}
\UseMicrotypeSet[protrusion]{basicmath} % disable protrusion for tt fonts
}{}
\IfFileExists{parskip.sty}{%
\usepackage{parskip}
}{% else
\setlength{\parindent}{0pt}
\setlength{\parskip}{6pt plus 2pt minus 1pt}
}
\usepackage{hyperref}
\hypersetup{
            pdftitle={Cibercultura},
            pdfauthor={Alex Ojeda Copa},
            pdfborder={0 0 0},
            breaklinks=true}
\urlstyle{same}  % don't use monospace font for urls
\usepackage{longtable,booktabs}
% Fix footnotes in tables (requires footnote package)
\IfFileExists{footnote.sty}{\usepackage{footnote}\makesavenoteenv{longtable}}{}
\usepackage{graphicx,grffile}
\makeatletter
\def\maxwidth{\ifdim\Gin@nat@width>\linewidth\linewidth\else\Gin@nat@width\fi}
\def\maxheight{\ifdim\Gin@nat@height>\textheight\textheight\else\Gin@nat@height\fi}
\makeatother
% Scale images if necessary, so that they will not overflow the page
% margins by default, and it is still possible to overwrite the defaults
% using explicit options in \includegraphics[width, height, ...]{}
\setkeys{Gin}{width=\maxwidth,height=\maxheight,keepaspectratio}
\setlength{\emergencystretch}{3em}  % prevent overfull lines
\providecommand{\tightlist}{%
  \setlength{\itemsep}{0pt}\setlength{\parskip}{0pt}}
\setcounter{secnumdepth}{5}
% Redefines (sub)paragraphs to behave more like sections
\ifx\paragraph\undefined\else
\let\oldparagraph\paragraph
\renewcommand{\paragraph}[1]{\oldparagraph{#1}\mbox{}}
\fi
\ifx\subparagraph\undefined\else
\let\oldsubparagraph\subparagraph
\renewcommand{\subparagraph}[1]{\oldsubparagraph{#1}\mbox{}}
\fi

% set default figure placement to htbp
\makeatletter
\def\fps@figure{htbp}
\makeatother

\usepackage{booktabs}
\usepackage{etoolbox}
\makeatletter
\providecommand{\subtitle}[1]{% add subtitle to \maketitle
  \apptocmd{\@title}{\par {\large #1 \par}}{}{}
}
\makeatother
\usepackage[]{natbib}
\bibliographystyle{apalike}

\title{Cibercultura}
\providecommand{\subtitle}[1]{}
\subtitle{Sílabo de la materia Cibercultura de la Carrera de Antropología de la UCB}
\author{Alex Ojeda Copa}
\date{2020-01-25}

\begin{document}
\maketitle

{
\setcounter{tocdepth}{1}
\tableofcontents
}
\hypertarget{introducciuxf3n}{%
\chapter*{Introducción}\label{introducciuxf3n}}
\addcontentsline{toc}{chapter}{Introducción}

La cultura no es una entidad estática, está en constante creación y recreación colectiva. Dependiendo de los espacios sociales involucrados, sus actores, interacciones y simbólos, se formaran dinámicas semióticas y prácticas culturales particulares. Habitamos ya espacios tradicionales de larga data como el hogar, las comunidades rurales, las ciudades, los mercados, las naciones, etc. Sin embargo, a finales del siglo XX e inicios del siglo XXI, como un corolario del proceso de globalización, aparece un nuevo espacio social muy peculiar que conecta diferentes territorios culturales: \emph{el ciberespacio}. Este nuevo espacio social, que se sostiene en las tecnologías digitales, representa un reto de comprensión para las ciencias sociales y da a lugar al complejo fenómeno de la \emph{cibercultura}.

Para la antropología, en particular, el ciberespacio representa dos oportunidades que son al mismo tiempo retos, uno metodológico y otro de mediación. Por lo general, en el estudio de fenómenos en el ciberespacio se utilizan métodos informáticos y cuantitativos asociados al \emph{big data}, que nos permiten ver patrones generales en un medio de abundante información; no obstante, aquellos métodos no permiten la profundidad cualitativa, aquí la antropología nos ofrece un valioso enfoque de \emph{small data}, que nos permita comprender la riqueza semiótica de esas dinámicas a partir de la etnografía digital. Al mismo tiempo, debido a que el ciberespacio tiene la capacidad de conectar diversos territorios culturales, nos encontramos con una constante interacción entre diversas culturas, experiencias y formas de conocimiento que no siempre es armónica, por lo que aquí se necesita una gestión del conocimiento colectivo y hasta una mediación intercultural. Así, la presente materia abordará la cibercultura tanto como un novedoso objeto de investigación antropológica como también una oportunidad para la gestión del conocimiento, el aprendizaje colectivo y la mediacion intercultural.

\hypertarget{historia-del-ciberespacio}{%
\chapter{Historia del ciberespacio}\label{historia-del-ciberespacio}}

\hypertarget{componentes-tecnoluxf3gicos}{%
\section{Componentes tecnológicos}\label{componentes-tecnoluxf3gicos}}

\begin{itemize}
\tightlist
\item
  El Internet

  \begin{itemize}
  \tightlist
  \item
    La red distribuida
  \item
    La conmutación de paquetes
  \item
    Arpanet\\
  \end{itemize}
\item
  La Web

  \begin{itemize}
  \tightlist
  \item
    La idea del hipertexto\\
  \item
    El Memex de Vannevar Bush
  \item
    El proyecto Xanadu de Ted Nelson\\
  \item
    El CERN\\
  \end{itemize}
\item
  La Web 2.0

  \begin{itemize}
  \tightlist
  \item
    Blogs
  \item
    Wikis
  \item
    Las redes sociales digitales\\
  \end{itemize}
\item
  El telefono inteligente

  \begin{itemize}
  \tightlist
  \item
    La pantalla tactil
  \item
    Las apps
  \end{itemize}
\end{itemize}

\hypertarget{componentes-culturales}{%
\section{Componentes culturales}\label{componentes-culturales}}

\begin{itemize}
\tightlist
\item
  El tropo de lo ``ciber''
\item
  Ciberpunk y otras narrativas
\item
  Manifiesto del ciberespacio
\item
  Comunidades virtuales
\item
  Identidades virtuales
\item
  Cibercultura como contracultura
\end{itemize}

\hypertarget{bibliografuxeda}{%
\section*{Bibliografía}\label{bibliografuxeda}}
\addcontentsline{toc}{section}{Bibliografía}

\begin{itemize}
\tightlist
\item
  Barlow, J. (1996). Declaración de independencia del ciberespacio. Periferica, 10, 241--242. \url{https://doi.org/10.25267/Periferica.2009.i10.22}
\item
  Castells, M. (2001). La galaxia internet. Madrid: Areté.
\item
  Castells, M., Fernández-Ardèvol, M., \& Linchuan Qiu, J. (2007). Comunicación móvil y sociedad. Barcelona: Ariel y Fundación Telefónica.
\item
  ISOC. (1997). Breve historia de internet. Internet Society. \url{https://www.internetsociety.org//es/internet/history-internet/brief-history-internet/}
\item
  Rheingold, H. (1993). The virtual community: Homesteading on the electronic frontier. Addison-Wesley.
\item
  Turkle, S. (1997). La vida en la pantalla: La construcción de la identidad en la era de Internet.
\item
  Turner, F. (2006). From counterculture to cyberculture. University of Chicago Press.
\end{itemize}

\hypertarget{teoruxedas-de-la-cibercultura}{%
\chapter{Teorías de la cibercultura}\label{teoruxedas-de-la-cibercultura}}

\begin{itemize}
\tightlist
\item
  La aldea global
\item
  Sociedad de la información
\item
  Sociedad red
\item
  Individualismo en red
\item
  Cibercultura
\item
  Etica Hacker
\item
  Cultura remix
\item
  Conectivismo
\item
  Capitalismo cognitivo y vigilancia
\item
  Antropología Cyborg

  \begin{itemize}
  \tightlist
  \item
    Automatas del siglo XVIII
  \item
    Robots
  \item
    Cyborgs
  \end{itemize}
\item
  Hiperculturalidad
\item
  Tecnoptimismo y tecnopesimismo
\end{itemize}

\hypertarget{bibliografuxeda-1}{%
\section*{Bibliografía}\label{bibliografuxeda-1}}
\addcontentsline{toc}{section}{Bibliografía}

\begin{itemize}
\tightlist
\item
  Ardévol, E. (2002). Cibercultura/ciberculturas: La cultura de Internet o el análisis cultural de los usos sociales de Internet. Actas del IX Congreso de Antropología de la Federación de Asociaciones de Antropolog'ia del Estado Español.
\item
  Ardèvol, E. (2003). La cibercultura: Un mapa de viaje; aproximaciones teóricas para un análisis cultural de Internet. Aportaciones al Seminario Pensar la Cibercultura: Antropología y Filosofía del Nuevo Mundo (Digital). Fundación Duques de Soria.
\item
  Bell, D. (2001). An Introduction to Cybercultures. Psychology Press.
\item
  Case, A. (2014). An illustrated dictionary of cyborg anthropology. CreateSpace Independent Publishing Platform.
\item
  Castells, M. (1997). La era de la información. Volumen 1: La sociedad red. Alianza editorial.
\item
  Escobar, A. (2005). Bienvenidos a Cyberia: Notas para una antropología de la cibercultura. Revista de estudios sociales, 22, 15--35.
\item
  Han, B.-C. (2014). En el enjambre. Herder Editorial.
\item
  Han, B.-C. (2018). Hiperculturalidad. Herder Editorial.
\item
  Haraway, D. J. (1995). Ciencia, cyborgs y mujeres: La reinvención de la naturaleza. Cátedra.
\item
  Himanen, P. (2015). La ética del hacker y el espíritu de la era de la información.
\item
  Lessig, L. (2005). Por una cultura libre: Cómo los grandes medios usan la tecnología y las leyes para encerrar la cultura y controlar la creatividad. Santiago: LOM Ediciones, 2005.
\item
  Lévy, P. (2007). Cibercultura: Informe al Consejo de Europa. Anthropos Editorial.
\item
  Mcluhan, M. (1996). Comprender los medios: Las extensiones del ser humano. Paidós.
\item
  Miller, D. (2016). How the world changed social media. \url{http://discovery.ucl.ac.uk/1474805/1/How-the-World-Changed-Social-Media.pdf}
\item
  Pariser, E. (2017). El filtro burbuja: Cómo la web decide lo que leemos y lo que pensamos. Taurus.
\item
  Webster, F. (2014). Theories of the information society. Routledge.
\end{itemize}

\hypertarget{pruxe1cticas-ciberculturales}{%
\chapter{Prácticas ciberculturales}\label{pruxe1cticas-ciberculturales}}

\hypertarget{ciberindividuo}{%
\section{Ciberindividuo}\label{ciberindividuo}}

\begin{itemize}
\tightlist
\item
  Identidad
\item
  Anonimidad
\item
  Soledad
\end{itemize}

\hypertarget{ciber-sociedad-civil}{%
\section{Ciber sociedad civil}\label{ciber-sociedad-civil}}

\begin{itemize}
\tightlist
\item
  Ciberasociaciones
\item
  Instituciones

  \begin{itemize}
  \tightlist
  \item
    Familia
  \item
    Juventud
  \item
    Educación
  \end{itemize}
\item
  Medios digitales
\item
  Violencia digital
\item
  Culturas indigenas y lo digital
\end{itemize}

\hypertarget{cibereconomuxeda}{%
\section{Cibereconomía}\label{cibereconomuxeda}}

\begin{itemize}
\tightlist
\item
  Trabajo digital
\item
  Comercio
\item
  Empresa
\end{itemize}

\hypertarget{ciberpoluxedtica}{%
\section{Ciberpolítica}\label{ciberpoluxedtica}}

\begin{itemize}
\tightlist
\item
  Ciudadanía digital
\item
  Regulación estatal
\item
  Libertad de expresión
\item
  Vigilancia
\item
  Movimientos sociales en red
\item
  Ciberactivismo
\item
  Ciberfeminismo
\item
  Hacktivismo
\item
  Gobierno electrónico y abierto
\end{itemize}

\hypertarget{cibertecnologuxeda}{%
\section{Cibertecnología}\label{cibertecnologuxeda}}

\begin{itemize}
\tightlist
\item
  Realidad virtual
\item
  Inteligencia artificial
\item
  Videojuegos
\end{itemize}

\hypertarget{bibliografuxeda-2}{%
\section*{Bibliografía}\label{bibliografuxeda-2}}
\addcontentsline{toc}{section}{Bibliografía}

\begin{itemize}
\tightlist
\item
  Cáceres, J. G. (1997). Comunidad virtual y cibercultura: El caso del EZLN en México. Estudios sobre las culturas contemporáneas, 3(5), 9--28.
\item
  Cáceres, J. G. (1998). Cibercultura, ciberciudad, cibersociedad hacia la construcción de mundos posibles en nuevas metáforas conceptuales. Estudios sobre las culturas contemporáneas, 4(7), 9--23.
\item
  Castells, M. (2009). Comunicación y poder (1. ed). Alianza.
\item
  Castells, M. (2012). Redes de indignación y esperanza: Los movimientos sociales en la era de internet. Alianza.
\item
  De Ugarte, D. (2007). El poder de las redes. David de Ugarte.
\item
  Dyson, L. E., Grant, S., \& Hendriks, M. (2015). Indigenous people and mobile technologies. Routledge.
\item
  Feixa, C. (2000). Generación@ la juventud en la era digital. Nómadas (Col), 13, 75--91.
\item
  González, G. M. (2010). De las culturas juveniles a las ciberculturas del siglo XXI. Revista Educación y ciudad, 18, 19--32.
\item
  Horst, H. A., \& Miller, D. (Eds.). (2012). Digital anthropology (English ed). Berg.
\item
  Jenkins, H. (2008). Convergence culture: La cultura de la convergencia de los medios de comunicación. Paidós.
\item
  Moya, M., \& Vázquez, J. (2010). De la Cultura a la Cibercultura: La mediatización tecnológica en la construcción de conocimiento y en las nuevas formas de sociabilidad. Cuadernos de antropología social, 31, 75--96.
\end{itemize}

\hypertarget{investigaciuxf3n-de-la-cibercultura}{%
\chapter{Investigación de la cibercultura}\label{investigaciuxf3n-de-la-cibercultura}}

\begin{itemize}
\tightlist
\item
  Big data vs.~small data
\item
  Etnografía digital

  \begin{itemize}
  \tightlist
  \item
    Diario de campo digital
  \item
    Análisis digital cualitativo
  \end{itemize}
\item
  Métodos virtuales
\item
  Métodos digitales
\end{itemize}

\hypertarget{bibliografuxeda-3}{%
\section*{Bibliografía}\label{bibliografuxeda-3}}
\addcontentsline{toc}{section}{Bibliografía}

\begin{itemize}
\tightlist
\item
  Amozurrutia, J. A. (2007). Cibercultur@ e iniciación en la investigación (Vol. 11). UNAM.
\item
  Cabrera, T. M., \& Cardona, J. J. C. (2014). La Etnografía: Una posibilidad metodológica para la investigación en cibercultura. Encuentros, 12(2), 93--103.
\item
  Del Fresno García, M. (2011). Netnografía. Editorial UOC.
\item
  Hine, C. (2011). Etnografía virtual. Editorial uoc.
\item
  Martínez Ojeda, B. (2006). Homo digitalis-etnografía de la cibercultura.
\item
  Rogers, R. (2013). Digital Methods. MIT Press.
\end{itemize}

\bibliography{book.bib,packages.bib}

\end{document}
